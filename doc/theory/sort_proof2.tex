\documentclass{article}
\usepackage{latexsym}
\usepackage{amsmath}
\usepackage{amsthm}
\usepackage{amsfonts}
\usepackage{amssymb}
\usepackage {mathpartir}
\usepackage{graphics}
\usepackage{color}
\usepackage{stmaryrd}
\usepackage{enumerate}

% special for mac OS !
\usepackage{graphicx}

%\usepackage[latin1]{inputenc}  % accents 8 bits dans le source
%\usepackage[T1]{fontenc}       % accents dans le DVI
\newcommand{\C}{\mathbb{C}}
\newcommand{\R}{\mathbb{R}}
\newcommand{\Q}{\mathbb{Q}}
\newcommand{\N}{\mathbb{N}}
\newcommand{\Z}{\mathbb{Z}}
\newcommand{\M}{\mathcal{M}}
\newcommand{\A}{\mathcal{A}}
\newcommand{\I}{\mathcal{I}}
\newcommand{\F}{\mathcal{F}}
\newcommand{\V}{\mathcal{V}}
\newcommand{\T}{\mathcal{T}}
\newcommand{\Pre}{\mathcal{P}}

\newcommand{\evaluation}[2][]{\ensuremath{\llbracket #2\rrbracket^{#1}}}


\newtheorem{definition}{Definition}
\newtheorem{theorem}{Theorem}
\newtheorem{corollary}[theorem]{Corollary}
\newtheorem{lemma}[theorem]{Lemma}
\DeclareMathOperator{\Null}{null}

\definecolor{LightGray}{gray}{0.8}
\definecolor{LightOrange}{rgb}{0.9, 0.8, 0.5}

\title{Why you can  drop sorts}
\author{}

\begin{document}
\maketitle


In this document, we will consider that there is not sub-sorting, no sort overloading, no parametric polymorphism, etc. but just a finite number of pairwise disjoint sorts. We are interested in the satisfiability/validity of a formula $\phi$  in multi-sorted logic over a given signature $\Sigma$. We claim that the satisfiability of $\phi$ is equivalent to the satisfiability of $\phi^{*}$ in the signature $\Sigma^{*}$, i.e. with all the sorts information losts.
The proof relies on the completeness of resolution : if there exist a derivation of the empty clause from $\phi^{*}$, it is possible to ``lift'' the derivation to the multi-sorted world. The key to this process is the fact that, in this context, if to terms are unifiable, then they are of the same sort.

\section{formalities}

\subsection{Tarski semantics for unisorted logic}

\begin{description}
\item[signature] A unisorted signature $\Sigma$ is given by :
\begin{itemize}
\item a set $\V$ of variables 
\item a set $\Pre$ of predicates, with their arity.
\item a set $\F$ of function symbols, with their arity.
\end{itemize}

\item[structure] A unisorted $\Sigma$-structure $\I$ is given by :
\begin{itemize}
\item a domain set $X$
\item a environment, which maps every variables of $\V$ to an element of the domain $X$.
\item for every predicate $P$ in $\Pre$, of arity $n$, a set of tuples included in $X^{n}$, defining the tuples on which $P$ is true.
\item for every function symbol $f$ in $\F$, of arity $n$, the function graph of $f$, that is a set of tuples included in $X^{n+1}$, such that for all $(x_1, \hdots, x_n)$, there exists one and only one $x$ such that  $(x_1, \hdots, x_n, x) \in f$ (which means that  $f(x_1, \hdots, x_n) = x$). 
\end{itemize}

Now, let's fix a a uni-sorted $\Sigma$-structure $\I$.

\item[interpretation of terms] A term is either a constant, a variable or a function symbol
\begin{itemize}                                                                                          
  \item[constants] $\evaluation{c}{\rho} = c \in X$
  \item[variables] if $x \in \V$, then $\evaluation{x}{\rho} = \rho(x)$
  \item[function symbols] if $f \in \F$, then $\evaluation{f(x_1,\ldots,x_n)}{\rho} = f(\evaluation{x_1}{\rho},\ldots,\evaluation{x_2}{\rho})$
\end{itemize}

\item[interpretation of formulas] 

\begin{itemize}
\item $\evaluation{P(x_1, \hdots, x_n)}{\rho} = (\evaluation{x_1}{\rho},\ldots,\evaluation{x_n}{\rho}) \in P$.
\item $\evaluation{\varphi_1 \land \varphi_2}{\rho} = \evaluation{\varphi_1}{\rho} \land \evaluation{\varphi_2}{\rho}$
\item $\evaluation{\varphi_1 \lor \varphi_2}{\rho} = \evaluation{\varphi_1}{\rho} \lor \evaluation{\varphi_2}{\rho}$
\item $\evaluation{� \varphi_1}{\rho} = � \evaluation{\varphi_1}{\rho}$
\item $\evaluation{\exists x. \varphi}{\rho} = \exists x \in X. \evaluation{\varphi}{\rho[x \to a]}$
\item $\evaluation{\forall x. \varphi}{\rho} = \forall x \in X. \evaluation{\varphi}{\rho[x \to a]}$
\end{itemize}

\end{description}


\subsection{Tarski semantics for multi-sorted logic}

We will consider the case of an arbitrary number of paiwise disjoint sorts $S_1, \hdots, S_n$. If $n=2$, we will call the sorts $O$ and $I$.

\begin{description}
\item[signature] A multi-sorted signature $\Sigma$ is given by :
\begin{itemize}
\item a set $\V$ of variable, along with their sort
\item a set $\Pre$ of predicates, with their types (i.e. the type of the tuples they accept). A type is an expresion of the form $s_1 * \hdots * s_k$, for a predicate of arity $k$. This means that the argument of $P$ must me a tuple in the set $S_{s_1} * \ldots * S_{s_k}$.
\item a set $\F$ of function symbols, with their types. The type of a function describes the sort of its arguments and of its result. It's an expression of the form  $s_1 * \hdots * s_k \to s_{k+1}$
\end{itemize}

\item[structure] A multi-sorted $\Sigma$-structure $\I$ is given by :
\begin{itemize}
\item for each sort, a domain set $S_i$
\item a well-sorted environment $\rho$, which map every variable with a element of the domain of the corresponding sort.
\item for every predicate $P$ in $\Pre$, of type  $s_1 * \hdots * s_k$, a set of tuples included in  $s_1 * \hdots * s_k$, defining the tuples on which $P$ is true.

\item for every function symbol $f$ in $\F$, of type $s_1 * \hdots * s_k \to s_{k+1}$, the function graph of $f$, that is a set of tuples included in $S_{s_1} * \hdots * S_{s_k} * S_{s_{k+1}}$.
\end{itemize}

Terms are interpreted the same way as in uni-sorted logics, except that the sort constraints hold.

\item[Interpretation of formulas] All the cases are similar to the uni-sorted case, except : 
\begin{itemize}
\item $\evaluation{\exists (x : S_u). \varphi}{\rho} = \exists x \in S_u. \evaluation{\varphi}{\rho[x \to a]}$
\item $\evaluation{\forall (x : S_u). \varphi}{\rho} = \forall x \in S_u. \evaluation{\varphi}{\rho[x \to a]}$
\end{itemize}

\end{description}

Note that we may want to consider an equality predicate per sort, to avoid equality overloading.

\section{Dropping sorts : the case without equality}

We are presenting here a weak version of the result, which holds only if sorts are not sharing any symbol. This means that we cannot have an overloaded equality predicate. In these condition, a simple remark can be made : terms that are unifiable are of the same sort. We are now proving this claim. This will allow use to show that forgetting sorts is sound with a purely syntactic argument. We first present rapidly the unification algorithm.

\subsection{the unification algorithm}

\begin{definition}
A \emph{unification problem} is a set of pair of terms of the form : $\Pre = \left\{ s_{1} \doteq t_{1}, \dots, s_{n} \doteq t_{n}\right\}$.
\end{definition}

\begin{definition}
$\sigma$ is a \emph{unifier} for a problem $\Pre$ if $\sigma(s_{i}) = \sigma(t_{i})$ for all contraint $s_{i} \doteq t_{i}$ in $\Pre$. 
\end{definition}

\begin{definition}
A problem $\Pre$ is in \emph{resolved form} iff it is of the form $\left\{ x_{1} \doteq t_{1}, \dots, x_{n} \doteq t_{n}\right\}$, where \begin{enumerate}
\item all the $x_{i}$ are pairwise distinct variables ($i \neq j \to x_{i} \neq x_{j}$).
\item $x_{i}$ does not appear in $t_{i}$ ($x_{i} \notin FV(t_{i})$).
\end{enumerate}
\end{definition}


\begin{definition}
Let $\Pre = \left\{ x_{1} \doteq t_{1}, \dots, x_{n} \doteq t_{n}\right\}$ be a problem in resolved form. The \emph{unifier associated with $\Pre$} is the substitution $\sigma_{\Pre} = \left\{ x_{1} \mapsto t_{1}, \dots, x_{n} \mapsto t_{n}\right\}$.
\end{definition}

We now define the unification algorithm as a set of rewriting rules.

\begin{mathpar}
 \colorbox{LightGray}{  \inferrule[Erase]{\Pre \cup \left\{ s \doteq s \right\} }{\Pre} } 
 \and
 \colorbox{LightGray}{  \inferrule[Orient]{\Pre \cup \left\{ t \doteq x \right\} \\ t \notin \V }{\Pre \cup \left\{ x \doteq t \right\} } }\and
 
 \colorbox{LightGray}{  \inferrule[Decompose]{\Pre \cup \left\{ f(s_{1}, \dots, s_{n}) \doteq f(t_{1}, \dots, t_{n}) \right\} }{\Pre \cup \left\{ s_{1} \doteq t_{1}, \dots, s_{n} \doteq t_{n} \right\} } } \and
  \colorbox{LightGray}{  \inferrule[Replace]{\Pre \cup \left\{ x \doteq s \right\} \\ x \in Var(\Pre) \\ x \notin FV(s)}{\Pre\left\{ x \mapsto s\right\} \cup \left\{ x \doteq s \right\} }}
 \end{mathpar}

These rules are both complete and sound : starting from a problem $\Pre$, repeatedly apply the rules. When it's no longer possible, if the resulting problem $\Pre'$ is in resolved form, then $\Pre$ is unifiable and $\sigma_{\Pre'}$ is a most general unifier for $\Pre$.

\subsection{Sorts and unification}

We want to prove that the mgu for a given problem preserves sorts.

% , where $s_{i} : s_{i}$ and $y_{i} : S_{i}$, for all $i$ (this means that the problem is \emph{well-sorted}). 

% (ie. if $t \doteq s$ appear in $\Pre$, then $t$ and $s$ are of the same sort)

\begin{lemma}
let $\sigma_{\Pre}$ be the most general unifier of a (unifiable) \emph{well-sorted} problem $\Pre = \left\{s_{1} \doteq t_{1}, \dots, s_{n} \doteq t_{n}\right\}$
Then for all term $t :s $, $\sigma_{\Pre}(t) : s$.
\end{lemma}

\begin{proof}
Since $\Pre$ is unifiable, the application of the rules of the unification algorithm will eventually lead to s problem $\Pre' = \left\{ x_{1} \doteq u_{1}, \dots, x_{m} \doteq u_{m}\right\}$ in resolved form.

We first prove that if a unification problem $\Pre$ is \emph{well-sorted} , then the application of a rule of the unification algorithm leads to a new well-sorted problem $\Pre'$. 

\begin{itemize}
\item The case of the \TirName{Erase} and \TirName{Orient} rules are straightforward.
\item The new constraints introduced by the \TirName{Decompose} rule are well-sorted : the original problem is well-sorted, so every argument of $f$ is compliant with the signature of $f$.
\item The \TirName{Replace} rule substitues an equality $x = s$ into $\Pre$. Since the original problem is well-sorted, $x$ and $s$ are of the same sort. Thus, $\Pre \left\{ x \mapsto s \right\}$ is well-sorted. 
\end{itemize}

Thus, by an immediate induction on the derivation of $\Pre'$ (in resolved form) from the original unification problem $\Pre$, $\Pre'$ is well-sorted.
$\sigma_{\Pre'}$ is then a \emph{sort-preserving} mgu of $\Pre$ (which maps variables to term of the same sort).

Another immediate induction on the structure of a well-sorted term $t$ gives us the final result.
\end{proof}


\subsection{Sorts and resolution}

We now use resolution to complete the argument. We show that if a resolution prov is found for an clause set with striped , then this proof is valid for the original clause set.

\begin{definition}
Let $\phi$ be a decorated formula in multi-sorted logic. Then $\phi^{*}$ is the same formula, in unsorted logic, without decorations. This notation is extended to clause sets and to signatures in a straightforward way.
\end{definition}

\begin{definition}
A clause set in unsorted logic over the signature $\Sigma^{*}$ is \emph{well sorted} if it is possible to assign sorts to all its variables, such that is becomes well-sorted with regard to $\Sigma$.
\end{definition}

Until the end of this section, we fix a signature $\Sigma$ for multi-sorted structures, and the corresponding $\Sigma^{*}$. All the formulas are considered with respect to these signatures.

This is the main result :

\begin{theorem}
$\phi$ is valid (w.r.t. $\Sigma$) if ond only if $\phi^{*}$ is valid (w.r.t. $\Sigma^{*}$).
\end{theorem}

%In order to prove this result, we will first state an intermediate lemma :

%\begin{lemma}
%Let $A$ and $A'$ be two (implicitly well-sorted) atoms in the multi-sorted world. Then $A$ and $A'$ are unifiable if and only if $A^{*}$ and $A'^{*}$ are (in the uni-sorted world).
%\end{lemma}

%\begin{proof}
%Since the only difference between $A$ and $A^{*}$ is the presence of sort decorations, the result is straightforward\ldots
%\end{proof}

Before completing the proof of theorem 2, let us recall the rules of binary resolution :

\begin{mathpar}
   \colorbox{LightGray}{  \inferrule[Resolution]{C \lor +A \\ C' \lor -A'}{C \sigma \lor C' \sigma} } \quad \colorbox{LightOrange}{$\sigma=mgu(A \doteq A')$}
   \and
   	\colorbox{LightGray}{  \inferrule[Factorisation]{C \lor +A \lor + A'}{C \sigma \lor +A \sigma} } \quad \colorbox{LightOrange}{$\sigma=mgu(A \doteq A')$}
\end{mathpar}

\begin{proof}[proof of theorem 2]
We will prove that the empty clause can be derived from $S$ using a finite number of resolution or factorisation steps if and only if it can be from $S^{*}$. As long as resolution is correct and complete in both the unsorted world and the multi-sorted world, this ensures theorem 2. 
For this purpose, we prove that if $S$ is a well-sorted set of clauses in uni-sorted logic, then the clause derived from $S$ by a resolution or factorisation step is well-sorted. This would allow us to conclude by a simple induction on the sequence of resolution/factorisation steps.

Here again, the multi-sorted $\Longrightarrow$ unsorted implication is trivial. We will now consider the case where we have a deduction step in the unsorted world, and we will show that it can be lifted to the multi-sorted world.

In both the cases of \TirName{Resolution} or \TirName{Factorisation}, assuming that the premises are well-sorted clauses and atoms, we want the conclusion and the unification problem to be well-sorted. Lemma 1 garantee than only the later is required, since it implies the former. And the unification problem is well-sorted, since $A^{*}$ and $A'^{*}$ are.
\end{proof}


\end{document}
